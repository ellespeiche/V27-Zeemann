\section{Diskussion}
\label{sec:Diskussion}

\FloatBarrier
\begin{table}
    \centering
    \caption{Berechnete Wellenlängenverschiebung aus.}
    \label{tab:v1}
    \sisetup{table-format=1.2}
    \begin{tabular}{c S[table-format=2.1] S[table-format=6.0] }
      \toprule
      %{$D$/cm}& \multicolumn{5}{c}{$U_{\symup{B}i} = \,\si{V}$}\\ überschrift über mehrere spalten
      %\midrule
       {Spektrallinie} & {$\lambda_{\si{D}}/\si{pm}$}& {$A$}\\
      \midrule
      \midrule
        rot & 48,9 & 209129 \\
        blau & 27,0 & 285458 \\
      \bottomrule
    \end{tabular}
\end{table}
\FloatBarrier

%Die prozentualen Fehler $f$ wurden mithilfe der Formel
%\begin{equation*}
%    f = \frac{\symup{\Delta}W}{W_{\symup{erm}}} 
%\end{equation*}
%bestimmt. Dabei ist $\symup{\Delta}W$ der ermittelte Fehler zum Wert $W_{\symup{erm}}$. 
%Für die Berechnung der Abweichungen $u$ wurde die Formel
%\begin{equation*}
%    u = \biggl| 1-\left(\frac{W_{\symup{erm}}}{W_{\symup{theo}}}\right) \biggr|
%\end{equation*}
%verwendet, wobei $W_{\symup{theo}}$ der Theoriewert ist.